\documentclass[a4paper,12pt]{article}

\usepackage{worksheet}  % Custom worksheet package (worksheet.sty must be in same directory)

% Metadata - set via worksheet.sty commands
\setsubject{Mathematics}
\setclass{8th}
\setworksheettitle{data handling}

\begin{document}

\makeworksheetheader

\begin{enumerate}

  % Question 1
  \item As a scientist studying a rare bird population, you have collected data over five years on nesting sites, egg survival rates, and chicks hatched. Which of the following visualizations would best help you show the population's overall trend (growth or decline) over these years?
  \equidistantoptions{pie chart showing the proportion of successful nests each year.}{scatter plot with the year on the x-axis and the number of chicks hatched on the y-axis.}{bar graph comparing the number of nesting sites each year.}{line graph plotting the egg survival rate over the five years.}

  % Question 2
  \item A local bakery wants to understand which of their new cookies are most popular. They recorded the number of each type of cookie sold each day for a week. What is the best way to organize this data and what type of graph would most effectively show the daily sales performance of each cookie, allowing for easy comparison?
  \equidistantoptions{Organize the data in a spreadsheet and create a bar chart comparing total sales of each cookie type for the week.}{Organize the data in a table showing daily sales for each cookie and use a line graph with each cookie type as a separate line to show trends over the week.}{Organize the data as a simple list of daily sales for all cookies combined and create a pie chart of the total weekly sales.}{Organize the data by grouping cookies sold on each day and create a scatter plot of cookie types versus days.}

  % Question 3
  \item Imagine you collected survey data on students' favorite after-school activities, and some students could select more than one activity. Which of the following is the MOST appropriate way to handle this 'multiple response' data to understand the most popular activities, and what is a common challenge with this type of data?
  \equidistantoptions{Count each student only once for their first choice and ignore other selections. A problem might be that it doesn't show the full picture of preferences.}{Create separate columns for each activity and mark '1' if a student chose it and '0' if not, then sum these columns to see overall popularity. A challenge is that it can create many columns if there are many activities.}{Ask every student to pick only one favorite activity to simplify the data. This avoids any issues with multiple choices.}{Calculate the average number of activities chosen by each student. This will directly show the most preferred activities and there are no issues.}

  % Question 4
  \item A school wants to hold a fundraising event and needs to choose a ticket price to maximize both attendance and revenue. They have past data showing how many people attended similar events at different ticket prices. What is the best way to use this data?
  \equidistantoptions{Look for the ticket price that had the highest attendance, regardless of the price.}{Calculate the total revenue (ticket price x attendance) for each past event and choose the price that resulted in the most revenue.}{Find the ticket price where attendance was average and revenue was also average.}{Ask the students what price they think is best for the tickets.}

  % Question 5
  \item Imagine you have two datasets for a city over a month: one showing the average daily temperature and the other showing daily ice cream sales. What is the most appropriate first step to explore the relationship between these datasets?
  \equidistantoptions{Calculate the average ice cream sales and compare it to the average temperature.}{Create a scatter plot with temperature on one axis and ice cream sales on the other to visually inspect for a pattern.}{Find the highest and lowest temperatures and see if they correspond to the highest and lowest ice cream sales.}{Calculate the total ice cream sales for the month and divide it by the total number of days.}

  % Question 6
  \item A class is studying the impact of screen time on sleep. They collected data on the number of hours students spend on screens per day and the average number of hours they sleep. To understand the relationship between these two variables, which statistical method would be most appropriate for analyzing this data and quantifying the connection?
  \equidistantoptions{Calculating the mean of screen time and the mean of sleep hours.}{Creating a scatter plot to visualize the data and calculating the correlation coefficient.}{Finding the mode of screen time and the mode of sleep hours.}{Performing a t-test to compare the average screen time and average sleep hours.}

  % Question 7
  \item You have a dataset of student test scores in Math and Science. If you want to compare the overall performance of the class in these two subjects, what type of chart would be most suitable?  If you then want to investigate if students who performed well in Math also tended to perform well in Science, how would your visualization likely change?
  \equidistantoptions{bar chart to compare average scores for Math and Science, then a scatter plot to show Math score vs. Science score.}{pie chart for Math scores and a line graph for Science scores.}{histogram for Math scores and a box plot for Science scores.}{Two separate pie charts, one for Math and one for Science.}

  % Question 8
  \item A small business owner has recorded their monthly expenses and profits for a year. To identify patterns and find areas for cost reduction, which of the following data processing and analysis steps would be most helpful?
  \equidistantoptions{Create a simple list of all expenses and profits without any calculation or grouping.}{Calculate the total profit for the year and compare it to the total expenses without looking at monthly trends.}{Create charts (like bar graphs or line graphs) to visualize monthly expenses and profits, calculate average monthly expenses, and identify months with unusually high expenses.}{Randomly pick a few months to examine their expenses in detail without considering the overall profit.}

  % Question 9
  \item Imagine you have collected data on the number of hours students slept each night and their self-reported energy levels on a scale of 1 to 10. Which type of graph would be most effective to visually explore if there's a relationship or pattern between hours of sleep and energy levels?
  \equidistantoptions{bar graph showing average energy levels for different ranges of sleep hours.}{scatter plot with hours of sleep on the x-axis and energy level on the y-axis.}{pie chart representing the distribution of energy levels.}{line graph connecting the energy levels of individual students in order of their sleep hours.}

  % Question 10
  \item A school conducted a survey on students' favorite fruits, broken down by grade level. To understand which fruits are most popular with different groups of students, which of the following approaches would be most helpful?
  \equidistantoptions{Calculating the average number of students who like each fruit across all grade levels and then choosing the top 3 fruits.}{Grouping the students by grade level, counting the number of students who prefer each fruit within each grade, and identifying popular fruits for each grade.}{Ignoring the grade level information and only looking at the total number of votes for each fruit.}{Making a bar graph of the total votes for each fruit and then randomly picking a few fruits to focus on.}

  % Question 11
  \item You have collected data on the number of hours students spent playing video games per week and their average grades in school. Which method would be most effective for visually exploring the potential relationship (correlation) between these two variables?
  \equidistantoptions{Creating a pie chart to show the proportion of students in different grade categories.}{Constructing a bar graph comparing the average hours of gaming for students in high-scoring versus low-scoring groups.}{Drawing a scatter plot with hours of video games on the horizontal axis and average grades on the vertical axis.}{Generating a frequency distribution table for the number of hours spent playing video games.}

  % Question 12
  \item A weather station records the amount of rainfall each month for a year. To best represent this data, showing the average monthly rainfall and clearly identifying months with unusually high or low rainfall, which of the following methods would be most effective?
  \equidistantoptions{line graph plotting monthly rainfall over time, with a horizontal line indicating the annual average rainfall.}{pie chart showing the proportion of total annual rainfall contributed by each month.}{simple bar chart displaying the rainfall for each month, ordered from highest to lowest.}{scatter plot of rainfall amount versus the number of days in the month.}

  % Question 13
  \item You have collected data on the number of steps individuals take each day and their self-reported mood on a scale from 1 (very unhappy) to 5 (very happy). To effectively explore if there's a relationship between daily physical activity (steps) and mood, which type of graph would be most appropriate?
  \equidistantoptions{bar chart comparing the average steps for each mood level.}{line graph showing the trend of steps and mood over a week.}{scatter plot with steps on the x-axis and mood on the y-axis.}{pie chart showing the proportion of days spent in each mood category.}

  % Question 14
  \item A school is conducting a survey on students' use of electronic devices. Many students reported using more than one type of device. Which of the following approaches is MOST appropriate for handling and visualizing this type of data?
  \equidistantoptions{Assign each student to a single primary device category based on frequency of use and then create a simple bar chart showing the count for each category.}{Create a stacked bar chart where each bar represents a student, and segments within the bar represent the different devices they use. The height of the bar would indicate the total number of devices used by that student.}{Use a Venn diagram to show the overlap in device usage between different categories (e.g., smartphones, tablets, laptops), and analyze the counts within each segment of the diagram.}{Calculate the percentage of students who use each device independently and present this information in separate pie charts for each device type.}

  % Question 15
  \item Imagine you have collected data on the number of students who participated in various sports teams at your school (e.g., basketball, soccer, track, volleyball). Which of the following methods would be MOST effective in visually highlighting the relative popularity of each sport based on participation numbers?
  \equidistantoptions{pie chart showing the percentage of students in each sport.}{line graph plotting the participation numbers over the last five years for each sport.}{bar graph with each sport on the x-axis and the number of participants on the y-axis.}{scatter plot comparing the number of participants in basketball against the number of participants in soccer.}

  % Question 16
  \item A local government has data on population density in different neighborhoods. How should they use this data to decide where to place new recycling bins?
  \equidistantoptions{Place bins in neighborhoods with the highest population density to serve the most people.}{Place bins in neighborhoods with the lowest population density to encourage more recycling.}{Place bins randomly across all neighborhoods regardless of population.}{Place bins only in neighborhoods with parks and public spaces.}

  % Question 17
  \item You have collected data on the number of minutes 8th-grade students spent on each question in a test, along with their final scores. Which type of visualization would be most effective to determine if how students pace themselves (time spent per question) affects their overall performance (final score)?
  \equidistantoptions{scatter plot with time spent on the x-axis and final score on the y-axis.}{bar chart showing the average time spent on each question.}{histogram of the students' final scores.}{pie chart illustrating the distribution of time spent across different question types.}

  % Question 18
  \item A marketing team has collected data showing the number of clicks on different online advertisements and the number of purchases made immediately after clicking. To determine which advertisements are most effective, what is the best way to analyze this data?
  \equidistantoptions{Calculate the total number of clicks for each ad and compare them.}{Calculate the total number of purchases for each ad and compare them.}{Calculate the conversion rate for each ad (purchases per click) and compare these rates.}{Find the average number of clicks and the average number of purchases across all ads.}

  % Question 19
  \item Imagine you have collected data on the number of hours of homework assigned for different subjects and the average test scores achieved in those subjects. Which of the following methods would be the most effective way to visually explore the relationship between the amount of homework assigned and student performance?
  \equidistantoptions{Creating a bar chart showing the total homework hours for each subject.}{Using a line graph to track the average test scores over a school year.}{Constructing a scatter plot with homework hours on one axis and average test scores on the other.}{Generating a pie chart to represent the proportion of different subjects taught.}

  % Question 20
  \item A scientist is investigating how different soil types affect plant growth. They measured the height of plants grown in three different soil types (Sandy, Loamy, and Clayey) over several weeks. To determine which soil type promotes the best growth, which of the following analytical approaches would be most appropriate?
  \equidistantoptions{Calculate the average plant height for each soil type and compare these averages.}{Create a scatter plot of plant height versus time for each soil type and look for the steepest slope.}{Perform a statistical test, such as an ANOVA, to see if there are statistically significant differences in plant heights among the soil types.}{List the maximum plant height achieved in each soil type and choose the soil with the highest maximum.}

  % Question 21
  \item Imagine you are a scientist studying the population of a rare bird. You have collected data over five years on its nesting sites, egg survival rates, and the number of chicks hatched. Which type of visualization would be most effective for showing trends in population growth or decline, and why?
  \equidistantoptions{bar chart showing the number of chicks hatched each year, as it clearly displays discrete values for each year and allows for easy comparison of population size changes.}{pie chart representing the percentage of successful nests each year, because it effectively shows the proportion of successful breeding events but doesn't directly illustrate population trends over time.}{scatter plot with time on the x-axis and the total number of chicks hatched on the y-axis, connected by a line, as this visually links consecutive years and highlights the overall trajectory of population change.}{line graph of egg survival rates over the five years, as this directly shows fluctuations in reproductive success which can be a key factor influencing population trends.}

  % Question 22
  \item A local bakery recorded the number of each of their new cookie types sold daily for a week. Which of the following options best describes how to organize this data to understand the popularity of each cookie and what type of graph would most effectively visualize the daily sales performance of each cookie type?
  \equidistantoptions{Organize by listing the total sales for each cookie type over the week and present the data using a pie chart to show the proportion of total sales for each cookie.}{Organize the data into a table showing daily sales for each cookie type and represent this using a stacked bar graph, where each bar represents a day and segments within the bar show the sales of individual cookie types.}{Organize the data in a spreadsheet with cookie types as rows and days as columns, and use a line graph with multiple lines (one for each cookie type) to track daily sales trends for each cookie.}{Organize the data by calculating the average daily sales for each cookie and display this using a pictograph, with symbols representing a certain number of cookies sold.}

  % Question 23
  \item You have survey data on students' favorite after-school activities, where some students selected more than one activity. Which of the following approaches is the MOST effective for understanding the overall popularity of individual activities, and what is a significant challenge you might face with this type of 'multiple response' data?
  \equidistantoptions{Calculate the percentage of students who selected each activity individually, and acknowledge that determining the absolute 'favorite' for each student becomes impossible.}{Count the total number of times each activity was selected, and recognize that this might overrepresent activities chosen by students who like many things.}{List all possible combinations of activities students chose, and ignore activities selected by fewer than 10% of students to simplify.}{Only consider students who chose exactly one activity to ensure a clear ranking, and discard the data from students with multiple responses.}

  % Question 24
  \item A school is planning a fundraising event and has historical data showing how many people attended similar events at different ticket prices. To suggest the best ticket price that aims to get the most attendees while also earning the most money, which of the following approaches would be most effective, and what is a key assumption needed for this analysis?
  \equidistantoptions{Plot the attendance data against ticket price and identify the price point where attendance is highest, assuming this directly correlates to maximum revenue.}{Create a revenue calculation for each historical price point (Revenue = Ticket Price × Attendance) and choose the price that yielded the highest revenue, assuming past trends will perfectly predict future attendance.}{Plot attendance versus ticket price, calculate the revenue for each point (Revenue = Ticket Price × Attendance), and then find the ticket price that corresponds to the peak of the revenue curve, assuming attendance will continue to decrease as price increases.}{Average the ticket prices and attendance numbers from all historical events and use this average to set the new ticket price, assuming a linear relationship between price and attendance.}

\end{enumerate}

% Answer Key (optional, commented out)
% \section*{Answer Key}
% Q1: B \\ 
% Q2: B \\ 
% Q3: B \\ 
% Q4: B \\ 
% Q5: B \\ 
% Q6: B \\ 
% Q7: A \\ 
% Q8: C \\ 
% Q9: B \\ 
% Q10: B \\ 
% Q11: C \\ 
% Q12: A \\ 
% Q13: C \\ 
% Q14: C \\ 
% Q15: C \\ 
% Q16: A \\ 
% Q17: A \\ 
% Q18: C \\ 
% Q19: C \\ 
% Q20: C \\ 
% Q21: C \\ 
% Q22: C \\ 
% Q23: B \\ 
% Q24: C \\ 


\end{document}
