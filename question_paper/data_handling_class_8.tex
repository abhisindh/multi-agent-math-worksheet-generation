\documentclass[a4paper,12pt]{article}

\usepackage{worksheet}  % Custom worksheet package (worksheet.sty must be in same directory)

% Metadata - set via worksheet.sty commands
\setsubject{Mathematics}
\setclass{Class 8}
\setworksheettitle{data handling}

\begin{document}

\makeworksheetheader

\begin{enumerate}

  % Question 1
  \item Imagine you are a scientist studying the local bird population. You've counted the number of robins in different parks for a week. Which of the following methods would be MOST helpful for organizing this data to easily see if there's a trend in their numbers over the week?
  \equidistantoptions{list of all the robin names you observed.}{table showing the date and the number of robins counted each day in each park.}{collection of random robin pictures from each park.}{spoken report of your daily observations without any written notes.}

  % Question 2
  \item Your school is collecting data on how students travel to school (bus, walk, car, bike). Which type of chart would best show the proportion of students using each mode of transport, and why?
  \equidistantoptions{A bar chart, because it easily compares the number of students for each category.}{A line graph, because it shows trends over time.}{A pie chart, because it visually represents parts of a whole, making proportions clear.}{A scatter plot, because it shows the relationship between two numerical variables.}

  % Question 3
  \item You have collected the heights of all the students in your class. To quickly understand the range of heights and identify the most common height, what steps would you take?
  \equidistantoptions{Calculate the average height and list all the individual heights.}{Find the difference between the tallest and shortest student and count the frequency of each height.}{Draw a bar graph and find the median height.}{Sum all the heights and sort the list from shortest to tallest.}

  % Question 4
  \item A local shop owner recorded the daily sales of ice cream for a month. Which of the following methods would be MOST useful for the owner to predict how much ice cream they might sell on a particularly hot summer day?
  \equidistantoptions{Averaging the sales from all the days in the month, regardless of the temperature.}{Focusing only on the sales from the hottest days in the month and finding the average of those sales.}{Looking at the sales from the coldest days in the month to see if demand decreases.}{Calculating the total sales for the entire month and dividing it by the number of days.}

  % Question 5
  \item You have the scores of 30 students on a math test. To understand the typical performance of the class, what is the correct way to calculate the average score, and what does this average represent?
  \equidistantoptions{Sum all 30 scores and divide by 30. This gives you the most frequent score in the class.}{Add all 30 scores together and then divide the total sum by 30. This result is the average score, which indicates the central tendency or typical score of the class.}{Find the highest and lowest scores and average them. This shows the range of scores in the class.}{Count how many students scored above 70 and divide by the total number of students. This tells you the proportion of high-achievers.}

  % Question 6
  \item Two classes, Class A and Class B, took the same test. Both classes have an average score of 75. However, the scores in Class A are spread out widely, while the scores in Class B are all very close to each other. What does this difference tell you about the learning in each class?
  \equidistantoptions{lass A has more consistent learning than Class B.}{lass B has more consistent learning than Class A.}{oth classes have the same level of learning.}{It is impossible to tell anything about learning from this information.}

  % Question 7
  \item Imagine you have a list of the daily temperatures for a month. Which of the following methods would be the best way to identify the warmest and coldest days from this data?
  \equidistantoptions{Calculate the average temperature of the month and see which days are closest to it.}{Sort the list of temperatures in ascending order to find the lowest temperature (coldest day) and sort in descending order to find the highest temperature (warmest day).}{Count how many times each temperature appears in the list.}{Add up all the temperatures and divide by the number of days in the month.}

  % Question 8
  \item A survey asked students about their favorite fruits. Which of the following ways to display the data would allow someone to instantly see which fruit is the most popular and which is the least popular?
  \equidistantoptions{A simple list of all the fruits mentioned.}{A bar graph showing the number of students who chose each fruit.}{A pie chart showing the percentage of students who chose each fruit.}{Both B and C}

  % Question 9
  \item You have collected data on the number of hours students spend on homework each night. To find the 'middle' value of this data, which statistical measure would you use and how would you calculate it?
  \equidistantoptions{}{}{}{}

  % Question 10
  \item You are designing a new game and want to ensure a level is challenging but not frustrating. You collect data on the number of attempts players take to beat a specific level. Which of the following data analysis methods would be most useful to determine the appropriate difficulty?
  \equidistantoptions{Calculate the average number of attempts and set the target completion time based on that average.}{Find the median number of attempts and observe the range of attempts to identify if there's a large cluster of players taking too many tries.}{Graph the number of players who beat the level after each attempt and identify the 'sweet spot' where a good portion of players succeed without excessive struggle.}{All of the above methods can be useful depending on the specific distribution of the data.}

  % Question 11
  \item Your class has collected data on the number of minutes students spend playing video games per week. If you wanted to compare the gaming habits of boys and girls, what would be the best way to organize and present this data?
  \equidistantoptions{table showing the average number of minutes for boys and girls, along with a bar graph comparing these averages.}{scatter plot with the total number of students on the x-axis and the total gaming minutes on the y-axis.}{pie chart showing the proportion of students who play video games versus those who don't, regardless of gender.}{frequency distribution table listing each student's name and their exact gaming minutes, then discussing patterns.}

  % Question 12
  \item A weather report provides the total annual rainfall (in millimeters) for a region over the last 10 years. To determine if there is a pattern of generally wetter or drier years, which of the following methods would be MOST useful?

A. Calculating the average rainfall for the entire 10-year period and comparing it to the rainfall of a single year.
B. Creating a line graph of the annual rainfall over the 10 years and observing the general trend.
C. Finding the maximum and minimum rainfall values recorded in the 10-year period.
D. Calculating the median rainfall for the 10-year period and seeing if it's close to the average.
  \equidistantoptions{}{}{}{}

  % Question 13
  \item Imagine you have a list of prices for various snacks at your school canteen: \$0.50, \$1.00, \$1.50, \$1.50, \$2.00, \$2.50, \$3.00. To understand the typical price of a snack, which measure would be most useful to calculate and what does it represent?

A. The median price, which represents the middle value when prices are ordered and is generally less affected by extreme prices than the mean.
B. The average price (mean), which is the sum of all prices divided by the number of snacks.
C. The mode price, which is the price that appears most frequently.
D. The range, which is the difference between the highest and lowest prices.
  \equidistantoptions{}{}{}{}

  % Question 14
  \item A company is tracking the number of customer complaints they receive each day. To identify if there are specific days of the week when problems are more likely to occur, which of the following data analysis methods would be most effective?
  \equidistantoptions{Calculating the average number of complaints per month.}{Creating a bar chart that shows the total number of complaints for each day of the week (Monday, Tuesday, etc.).}{Finding the median number of complaints across all days.}{Counting the total number of complaints over a year.}

  % Question 15
  \item Sample question on You have a dataset of the ages of people attending a community event. How would you describe the 'spread' or 'variability' of the ages in the group? (LaTeX: $x^2 + y^2$)
  \equidistantoptions{: Option 1}{: Option 2}{: Option 3}{: Option 4}

  % Question 16
  \item Sample question on Imagine you are analyzing the results of a science experiment where you tested different fertilizers on plant growth. How would you present your findings to show which fertilizer was most effective? (LaTeX: $x^2 + y^2$)
  \equidistantoptions{: Option 1}{: Option 2}{: Option 3}{: Option 4}

  % Question 17
  \item Sample question on You are given a dataset of the number of books borrowed from the library each day. How could you use this data to estimate how many books might be borrowed on a Saturday? (LaTeX: $x^2 + y^2$)
  \equidistantoptions{: Option 1}{: Option 2}{: Option 3}{: Option 4}

  % Question 18
  \item Sample question on A town council is looking at data on the amount of recycling collected each week. How could they use this to set realistic recycling targets for the future? (LaTeX: $x^2 + y^2$)
  \equidistantoptions{: Option 1}{: Option 2}{: Option 3}{: Option 4}

  % Question 19
  \item Sample question on You have a set of scores for a spelling bee. How would you identify outliers – scores that are unusually high or low – and what might those outliers represent? (LaTeX: $x^2 + y^2$)
  \equidistantoptions{: Option 1}{: Option 2}{: Option 3}{: Option 4}

  % Question 20
  \item Sample question on A sports team tracks the number of goals scored by each player. How could this data be used to decide on team strategy or identify player strengths? (LaTeX: $x^2 + y^2$)
  \equidistantoptions{: Option 1}{: Option 2}{: Option 3}{: Option 4}

  % Question 21
  \item Sample question on You are given a dataset of the population growth of a city over several decades. How could you use this data to make a prediction about the city's population in the next 20 years? (LaTeX: $x^2 + y^2$)
  \equidistantoptions{: Option 1}{: Option 2}{: Option 3}{: Option 4}

  % Question 22
  \item Sample question on A museum is collecting data on the number of visitors each day. How would they use this data to plan staffing levels? (LaTeX: $x^2 + y^2$)
  \equidistantoptions{: Option 1}{: Option 2}{: Option 3}{: Option 4}

  % Question 23
  \item Sample question on You have a list of the distances students live from school. How would you calculate the average distance and what does this average tell you about the commute of the students? (LaTeX: $x^2 + y^2$)
  \equidistantoptions{: Option 1}{: Option 2}{: Option 3}{: Option 4}

  % Question 24
  \item Sample question on A scientist is studying the impact of sunlight on plant growth. They have data on the height of plants grown in different amounts of light. How would they visually represent this to show the relationship? (LaTeX: $x^2 + y^2$)
  \equidistantoptions{: Option 1}{: Option 2}{: Option 3}{: Option 4}

  % Question 25
  \item Sample question on You are given data on the number of hours a group of people slept each night. How would you find the mode and what does the mode tell you about the most common sleep duration? (LaTeX: $x^2 + y^2$)
  \equidistantoptions{: Option 1}{: Option 2}{: Option 3}{: Option 4}

\end{enumerate}

% Answer Key (optional, commented out)
% \section*{Answer Key}
% Q1: B \\ 
Q2: C \\ 
Q3: B \\ 
Q4: B \\ 
Q5: B \\ 
Q6: B \\ 
Q7: B \\ 
Q8: D \\ 
Q9: B \\ 
Q10: D \\ 
Q11: A \\ 
Q12: B \\ 
Q13: A \\ 
Q14: B \\ 
Q15: A \\ 
Q16: A \\ 
Q17: A \\ 
Q18: A \\ 
Q19: A \\ 
Q20: A \\ 
Q21: A \\ 
Q22: A \\ 
Q23: A \\ 
Q24: A \\ 
Q25: A \\ 


\end{document}
