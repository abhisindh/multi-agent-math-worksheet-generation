\documentclass[a4paper,12pt]{article}

\usepackage{worksheet}  % Custom worksheet package (worksheet.sty must be in same directory)

% Metadata - set via worksheet.sty commands
\setsubject{Mathematics}
\setclass{8th grade}
\setworksheettitle{DATA Handling}

\begin{document}

\makeworksheetheader

\begin{enumerate}

  % Question 1
  \item Imagine you have collected the heights of all students in your 8th-grade class. Which method would best represent this data to show the most common height range?
  \equidistantoptions{A pie chart showing the proportion of students in each height category.}{A bar graph where the x-axis shows different height ranges and the y-axis shows the number of students in each range.}{A line graph connecting the heights of students in ascending order.}{A scatter plot with student names on the x-axis and their heights on the y-axis.}

  % Question 2
  \item Your school is planning to organize a sports day. You have data on the number of students participating in various sports events last year. To predict which sports events might be most popular this year based on last year's participation, which type of graph would be most helpful and why?
  \equidistantoptions{A bar graph, because it clearly shows the number of participants for each individual event, making comparisons easy.}{A line graph, because it is best for showing trends over time.}{A pie chart, because it shows the proportion of students participating in each event relative to the total.}{A scatter plot, because it is useful for showing the relationship between two different variables.}

  % Question 3
  \item Imagine a class of 20 students took a test, and their scores were all entered correctly.  However, one score was accidentally entered as 100 instead of the actual score of 10.  How would this single incorrect entry (an outlier) most likely affect the average (mean) score and the median score of the class?
  \equidistantoptions{The average score would increase, and the median score would stay the same.}{The average score would decrease, and the median score would increase.}{oth the average score and the median score would increase.}{The average score would increase significantly, while the median score would change very little.}

  % Question 4
  \item A local shop owner wants to know the most popular ice cream flavor from the past month's sales. Which of the following methods would be the BEST way to analyze this data and present the findings?
  \equidistantoptions{Create a bar graph showing the sales of each flavor, with the tallest bar representing the most popular flavor.}{Calculate the average sales of all flavors and see which flavor is closest to the average.}{List all the flavors in alphabetical order and highlight the one that appears first.}{Sum the number of scoops sold for each flavor and then pick the flavor with the highest sum.}

  % Question 5
  \item Imagine you collected survey data about your friends' favorite video games. Which type of visual representation would best help you quickly identify the most popular game at a glance?
  \equidistantoptions{pie chart, showing the proportion of friends who prefer each game.}{line graph, plotting the popularity of games over time.}{scatter plot, comparing the number of friends who like each game against another variable.}{bar graph, with each bar representing a game and its height showing how many friends chose it as their favorite.}

  % Question 6
  \item You are designing a simple survey for your classmates to find out their favorite type of music. Which of the following questions would give you the clearest and most usable data for this purpose?
  \equidistantoptions{What are your thoughts on music?}{Do you like any music at all?}{Please list your top 3 favorite music genres (e.g., Rock, Pop, Hip-Hop, Classical, Electronic).}{Is music good or bad?}

  % Question 7
  \item Suppose you recorded the daily temperatures for a week. To find the average temperature for that week, you would:

A. Add all the daily temperatures and divide by 7.
B. Add all the daily temperatures and divide by the highest temperature.
C. Add all the daily temperatures and divide by the number of days it rained.
D. Find the highest and lowest temperatures and divide by 2.
  \equidistantoptions{}{}{}{}

  % Question 8
  \item A scientist is observing plant growth over several weeks. To effectively compare the growth rates of plants under different conditions, which type of graph would be most suitable?
  \equidistantoptions{bar graph showing the final height of each plant at the end of the experiment.}{pie chart showing the proportion of plants that grew successfully.}{line graph with time on the x-axis and plant height on the y-axis, with separate lines for each condition.}{scatter plot showing the height of plants versus the amount of sunlight they received.}

  % Question 9
  \item Sample question on You're planning a class party and need to decide how many pizzas to order. You have data on how many slices students typically eat. How would you use this data to make an informed decision? (LaTeX: $x^2 + y^2$)
  \equidistantoptions{: Option 1}{: Option 2}{: Option 3}{: Option 4}

  % Question 10
  \item Sample question on Consider the data: 5, 10, 15, 20, 25. If you add the number 50 to this set, how does the mean change compared to the median? (LaTeX: $x^2 + y^2$)
  \equidistantoptions{: Option 1}{: Option 2}{: Option 3}{: Option 4}

  % Question 11
  \item Sample question on Imagine you have data on the number of hours students spend on homework each day. How would you create a frequency table to organize this information? (LaTeX: $x^2 + y^2$)
  \equidistantoptions{: Option 1}{: Option 2}{: Option 3}{: Option 4}

  % Question 12
  \item Sample question on Your teacher asks you to present the class's favorite colors. What are the advantages of using a bar graph versus a pie chart for this data? (LaTeX: $x^2 + y^2$)
  \equidistantoptions{: Option 1}{: Option 2}{: Option 3}{: Option 4}

  % Question 13
  \item Sample question on You have data about the number of siblings each student in your class has. How can you calculate the mode of this data, and what does it represent? (LaTeX: $x^2 + y^2$)
  \equidistantoptions{: Option 1}{: Option 2}{: Option 3}{: Option 4}

  % Question 14
  \item Sample question on A company is launching a new product. They conduct a survey about potential customer interest. What are some potential biases that might affect the survey results? (LaTeX: $x^2 + y^2$)
  \equidistantoptions{: Option 1}{: Option 2}{: Option 3}{: Option 4}

  % Question 15
  \item Sample question on You have data on the ages of people attending a movie. How would you organize this data to understand the age group that attends the most? (LaTeX: $x^2 + y^2$)
  \equidistantoptions{: Option 1}{: Option 2}{: Option 3}{: Option 4}

  % Question 16
  \item Sample question on Design a simple experiment to collect data on how different amounts of sunlight affect plant growth. What variables would you measure? (LaTeX: $x^2 + y^2$)
  \equidistantoptions{: Option 1}{: Option 2}{: Option 3}{: Option 4}

  % Question 17
  \item Sample question on You are given a set of data points for a science experiment. How would you identify any data points that seem unusual or incorrect (outliers)? (LaTeX: $x^2 + y^2$)
  \equidistantoptions{: Option 1}{: Option 2}{: Option 3}{: Option 4}

  % Question 18
  \item Sample question on Imagine you have data on the number of books borrowed from the library each day. How would you calculate the range of this data? (LaTeX: $x^2 + y^2$)
  \equidistantoptions{: Option 1}{: Option 2}{: Option 3}{: Option 4}

  % Question 19
  \item Sample question on Your class is discussing environmental issues. You have data on the amount of plastic waste collected from different areas. How would you present this data to show which area has the most plastic waste? (LaTeX: $x^2 + y^2$)
  \equidistantoptions{: Option 1}{: Option 2}{: Option 3}{: Option 4}

  % Question 20
  \item Sample question on You are given a dataset of student scores on a math quiz. How can you calculate the mean, median, and mode? What does each of these measures tell you about the class's performance? (LaTeX: $x^2 + y^2$)
  \equidistantoptions{: Option 1}{: Option 2}{: Option 3}{: Option 4}

  % Question 21
  \item Sample question on A sports coach has data on the running times of athletes. How can they use this data to identify the fastest runners and potential areas for improvement? (LaTeX: $x^2 + y^2$)
  \equidistantoptions{: Option 1}{: Option 2}{: Option 3}{: Option 4}

  % Question 22
  \item Sample question on You have data on the daily rainfall for a month. How would you create a line graph to show the trend of rainfall over time? (LaTeX: $x^2 + y^2$)
  \equidistantoptions{: Option 1}{: Option 2}{: Option 3}{: Option 4}

  % Question 23
  \item Sample question on Imagine you have data on the number of minutes students spend playing video games each day. How would you group this data into intervals (e.g., 0-30 minutes, 31-60 minutes) to make it easier to analyze? (LaTeX: $x^2 + y^2$)
  \equidistantoptions{: Option 1}{: Option 2}{: Option 3}{: Option 4}

  % Question 24
  \item Sample question on Your school is considering changing the lunch menu. They have data on student preferences for different food items. How would you analyze this data to recommend the most popular options? (LaTeX: $x^2 + y^2$)
  \equidistantoptions{: Option 1}{: Option 2}{: Option 3}{: Option 4}

  % Question 25
  \item Sample question on You are given a set of numbers representing the ages of people in a room. If you sort the numbers, how do you find the median? (LaTeX: $x^2 + y^2$)
  \equidistantoptions{: Option 1}{: Option 2}{: Option 3}{: Option 4}

\end{enumerate}

% Answer Key (optional, commented out)
% \section*{Answer Key}
% Q1: B \\ 
Q2: A \\ 
Q3: D \\ 
Q4: D \\ 
Q5: D \\ 
Q6: C \\ 
Q7: A \\ 
Q8: C \\ 
Q9: A \\ 
Q10: A \\ 
Q11: A \\ 
Q12: A \\ 
Q13: A \\ 
Q14: A \\ 
Q15: A \\ 
Q16: A \\ 
Q17: A \\ 
Q18: A \\ 
Q19: A \\ 
Q20: A \\ 
Q21: A \\ 
Q22: A \\ 
Q23: A \\ 
Q24: A \\ 
Q25: A \\ 


\end{document}
